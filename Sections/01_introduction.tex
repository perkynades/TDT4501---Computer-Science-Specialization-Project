\section{Introduction}

\textcolor{red}{Legge til preface}

\textcolor{red}{Legge til abstract}

\textcolor{red}{Skrive conclusion}

\subsection{Motivation}
In software development, listening to the users through usability tests is important. It allows product teams to gain a more realistic picture of the application they are working on, increasing quality. It also increases transparency with their users, making the users feel more heard. 

The IT department of the Norwegian Labour and Welfare Administration (NAV) is responsible for developing and running NAV's IT-systems.The product-teams at NAV IT are organized as autonomous and are mostly working on IT-systems for serving benefits and social-services to the citizens of Norway. The product-teams are organized cross-functionally, where most teams include: a software developer, a designer and, a legal advisor. Most of the product-teams follows the principles of Agile with rapid feedback each IT-system the product-teams are developing, which includes feedback on the usability of each particular IT-system. To gain usability feedback it is necessary to meet NAV's users by performing usability tests. However it is an increasing challenge for the product-teams to meet users for usability tests. A part of this is that there is no standardisation of tools used for the recruitment of candidates for usability tests. This has lead to NAV IT starting the initiative of Innbyggerpanelet, a digital solution for recruitment of candidates for usability tests.

However even if you have a digital solution that makes the management of recruiting candidates simpler, related papers suggest its a challenge to make candidates participate a usability test. From a user base of 200 users, 52 consented to participate in a study, while only 9 participants (17\%) completed the study [\cite{pkf_2018}]. Improper incentive might also lead to the participants registered in such a system not completing the study due to them feeling like its not worth their time and effort to participate [\cite{pkf_2018}]. These factors might suggest that digital solutions like Innbyggerpanelet need a large user-base before it can be of value to the product teams at NAV.

As comparable to clinical trials, NAV's studies calls for participants with a particular set of criteria, such as age, gender, life status, job status etc. This can exclude a large portion of the participants available in Innbyggerpanelet, slowing down recruitment and preventing new participants to be included in the studies [\cite{nc_2020}]. This means that the criteria for participants collected by Innbyggerpanelet needs to be specific enough to be helpful, while not being too specific, excluding a large amount of potential participants.

Newly employed designers at NAV IT have also stated that performing usability tests is different at NAV/public sector then private sector, after performing usability tests in the private sector. Also stating that the newly employeed designers do not know how usablity testing is performed at NAV IT/public sector, thus not knowing the challenges of performing usability tests in the Norwegian public sector. There has also been a lack of focus on the quality of usability of NAV's IT-systems. Usability testing is something that "steals" resources from what the quality of the IT-systems are measured upon [NAV presentation, User-meetings as of february 2022]. This lack of focus might mean that there are undescovered issues in usability testing, due to the lack of performing usability testing.

\textcolor{red}{her må jeg evt utdype mer om hvorfor usability testing "stjeler" ressurser}

\subsection{Project Goals and Research Questions}
The intended goal of this research is to gain a deeper understanding of what are the challenges of performing usability tests in the Norwegian public sector. It also intends to understand if proposed solutions such as Innbyggerpanelet actually are solutions to aid in usability testing, or just attacks symptoms of challenges in recruitment of candidates for usability tests at in the Norwegian public sector. Conducting a literature review on these topics will give a basis for answering the following research questions: 

\begin{itemize}
    \item \textbf{RQ1:} What are the challenges of performing usability tests in the Norwegian public sector?
    \item \textbf{RQ2:} What are the specific challenges in recruitment of qualified candidates to usability tests in the Norwegian public sector?
\end{itemize}


\subsection{Problem owner}
This study is based on a real problem to be solved through literature review. The problem is mainly relevant for NAV and the designers and design researchers working at NAV, however it can still be relevant for other sectors in the Norwegian public sector.

NAV (Norwegian Labour and Welfare Administration) is the main provider of social services to the citizens of Norway. From 2014 NAV has had an increased focus on improving their digital solutions, in turn providing better services to the citizens of Norway. Part of this focus has been on improving the usability of their digital solutions, as this has been an concurrent issue. The most common method to asses the usability of something is by conducting a usability test, which also NAV practices at their own test-lab. Increased usability is also stated in NAV's three year plan (2022-2025): "The contents of nav.no is easy to understand so that it's users can maintain their rights and duties" [NAV presentation, Business strategy and three-year priorities].

The author of this report also have a shared interest in its findings. The author works as a software developer at NAV, with the aim of creating supporting solutions for the designers and design researchers at NAV.

\subsection{Contributions}
The main contribution of this report is the increased knowledge of potential problems and fallpits that the designers and design researchers at NAV could encounter when performing usability tests. This can then give increased benefits to the Norwegian citizens as NAV is able to provide more usable digital solutions. This knowledge could also be extended beyond NAV and other design communities, as usability testing is a common practice for determining usability.

\subsection{Report outline}

\textcolor{red}{Snakke om hva som kommer (skrive denne når rapporten er ferdig) }