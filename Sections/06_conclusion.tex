\section{Conclusion}
It is an increasing challenge for the actors in the Norwegian public sector to meet users for usability tests and meet users with the correct qualifications. These challenges stems from a lack of standardisation in the recruitment of users to usability tests and has led to NAV IT starting the initiative of Innbyggerpanelet. There has also been a lack of focus on the measuring of usability of NAV IT's solutions, leading to potential challenges in the performing of usability testing in the norwegian public sector being undiscovered. In this report the mentioned challenges have been studied through an literature review.

The literature review found several challenges in usability testing. Usability testing as a tool can be an issue when used in an agile workflow and usability testing is not also considered properly as the tool to use. Choosing the incorrect amount of participants for a usability test can lead to usability errors being undiscovered. Cognitive bias can also lead to a participant in a usability test not acting naturally in the interaction with a IT-system.

Several challenges in the recruitment of participants for usability tests were also uncovered. Monetary incentives can result in the potential participant refusing to consent to a usability test if not proper. If logistical factors for the participant is not considered properly, it can lead to the potential participant not consenting to the usability test. Considerate interaction between potential participants and researchers can be vital in the recruitment of the potential participant. Marketing is important in the outreach of potential participants for usability tests. Crowdsourcing could have both positive and negative effects in the recruitment of participants for usability testing.

These results do have some limitations. Only 1 paper of the literature review had the context of the public sector, leading to the found results might not being relevant for the public sector. Another relevant limitation could be that only 1 paper discusses recruitment as a challenge for usability testing, which runs the risk of the other papers on recruitment are not relevant for usability testing.

It can therefore be some validity issues on if the research questions of this paper have been answered, thus further research should be conducted on the challenges in usability testing and recruitment to usability testing with the context of the public sector.
