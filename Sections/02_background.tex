\section{Background}

\subsection{Usability}
According to the ISO 9241-11 standard, usability is the extent to which a product can be used by specified users to achieve specified goals with effectiveness, efficiency, and satisfaction in a specified context of use [\cite{iso_2022}]. Poor usability can lead to unexpected consequences such as the user not achieving a goal or achieving the wrong goal [\cite{iso_2022}].

\subsection{Usability Testing}
As described in the authors previous work [\cite{ano_2021}], usability testing is a UX research methodology. The goal of usability testing is to identity problems in the design of a product or service, uncover opportunities to improve or learn about the target user’s behaviour and preferences.

In a usability testing session, a researcher (called a “facilitator” or a “moderator”) asks a participant to perform tasks, usually using one or more specific user interfaces. While the participant completes each task, the researcher observes the participant’s behaviour and listens for feedback [\cite{ano_2021}].

\textcolor{red}{se på den over mer senere, skjønte ikke helt veileders tilbakemelding}

The facilitator or moderator administrates the usability test and tasks for the participant. As the participants are performing the tasks, the facilitator observers the behaviour of the participant and listens for feedback. The facilitator should ensure the test result is high-quality, valid data, without accidentally influencing the participant [\cite{ano_2021}].

The participant of a usability test should be a realistic user of the product or service being studied. The participant might also have a similar background to the target user group, or might have the same needs. The participants are encouraged to think out loud during the testing. The goal of this approach is to understand the participants behaviours, goals, thoughts, and motivations [\cite{ano_2021}].

The tasks in a usability test program should be very specific and something a realistic user of the product often will perform or more open for the user to explore. Defining which elements to test for by importance and prioritisation can be done by using the 7 steps model: (1) Determine the most important user task. (2) Discover which system aspects are of most concern. (3) Group items from 1 \& 2, then sort issues by importance to users and organisation. (4) For each top issue, condense the information into a problem statement. (5) For each problem statement, list research goals. (6) For each research goal, list participant activities and behaviours, and. (7) For each group of goals, write user scenarios [\cite{ano_2021}].  

\subsection{Agile Software Development}
As described by Atliassian, agile is an iterative approach to project management and software development that helps teams deliver value to their customers faster and with fewer headaches. Instead of betting everything on a "big bang" launch, an agile team delivers work in small, but consumable, increments. Requirements, plans, and results are evaluated continuously so teams have a natural mechanism for responding to change quickly [\cite{a_2022}]. 

\subsection{Public Sector}
A public sector is an entity which include public goods and governmental services such as military, infrastructure, public education and, health care [\cite{ps_2022}]. The public sector differs from the private sector, as it does not concern privately owned organisations, nonprofit organisations or households [\cite{os_2022}]. 

\subsubsection{Digitalisation in the Norwegian public sector}
The government of Norway has in the last decades has an increased focus on digitalisation of the Norwegian public sector with the most recent digitization strategy from 2019-2025 [\cite{r_2019}]. The main goal of this digitalisation strategy is for different actors in the Norwegian public sector to co-opperate by the development of a shared digital eco-system. The goal of this development project is to aid the development of user-centered system development and a more effective and coordinated exploitation of the Norwegian public sectors IT-systems [\cite{r_2019}].