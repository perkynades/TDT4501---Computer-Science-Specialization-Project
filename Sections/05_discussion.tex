\section{Discussion}

\subsection{RQ1: What are the challenges of performing usability tests in the Norwegian public sector?}
Including the right amount of participants for usability testing in the public sector is something that needs to be chosen carefully. As described in \autoref{sec:participants-in-usability-testing} the number of participants needed to discover usability errors is between 5-20 [\cite{ola_2019}]. As the number of participants of each usability test cycle increases, so does also project time and costs. This is important for UX professionals in the public sector to consider, as projects in the public sector are usually bound to strict budgets. Meaning that UX professionals have to balance between participants needed to find usability problems and how much finding each usability problem will cost.

A countermeasure to the "cost per usability problem" increasing when including more participants is to instead increase task coverage [\cite{gl_2007}]. There would then not be the same need for an increased amount of participants, as the task coverage would detect the same usability problems as increased amount of participants. However this focus can lead to the usability problems that important for the public sector not to be discovered, as it only focuses on the amount of participants, not the qualification of participants. IT-systems in the public sector, especially in the case of NAV, usually have the purpose of delivering some sort of benefit or social-service. Is is therefore important to do usability testing with participants who are eligible to those benefits or social-services, in order to detect usability problems that is most likely effect those participants. This also applies to public sector services that span nation wide, as with the case Norway, 15\% of the population has some sort of impaired functioning [\cite{nhf_2022}].

How a usability test is performed for testing the usability of a IT-solution in the public sector is something which should be considered with care. As described in \autoref{sec:problems-during-a-usability-test}, the participant might think that the moderator of the usability test is the one who creating the system [\cite{dn_2016}]. This leading to the participant blaming system faults on themselves. Participants usability testing the solution for a social-service might be in the application process or is receiving that social service. The participant could then act in a way as to not affect either the application process or the receivable of social services in a negative way, even though the results of the usability test have no effect on this. Also using a participant already applying or receiving a specific social-service in usability testing can be challenging, as the participant would already be familiar with the system. This can lead to the system being interpreted as more easy-to-use than in reality, and potentially uncovering less usability problems.

As national budgets fluctuate and restricts what focus the development of public sector IT-systems should have, it is important that usability testing does not introduce new challenges. If the usability test is incorrectly applied it an lead to poor ideas being promoted, and leading the development of the IT-system being directed in the wrong direction. It can also lead to the developers of the IT-system not focusing on the important problems, as they were not discovered by the testing. These issues can lead to increased costs in the development of the IT-systems. It can also end up making the user feel distrust with the public sector, as the IT-systems are not providing what they are entitled to, because the testing was faulty.  

As IT development in the public sector moves away from large infrequent deliveries, to a more iterative workflow with small rapid deliveries [\cite{at_2021}], some new challenges for usability testing has been discovered. These new iterative workflows has forced UX professionals to fewer usability tests, which runs the risk of serious usability issues not being discovered. This has also lead to usability tests being of smaller scale and including less tasks than normally, which could lead to serious usability problems of IT-systems in the public sector not being discovered. The purpose of the IT-systems in the public sector is complex with many nuances, however the iterative workflows usually focus on a single feature. The usability tests would then have to be altered to only focus on a single feature, which could lead to usability problems not being detected, as the participants is not testing the big picture of the system [\cite{ln_2012}].

The main purpose of IT-system in the public sector is to serve the server the population of a country with benefits and other social services. And its users are usually dependant on these services in their everyday life, meaning that the services providing these benefits and social services need to be useful. The use of usability testing can therefore be an issue as it focuses on the "to hard to use" aspects of a system, not how good it is to serve it's purpose [\cite{sg_2008}]. Usability testing a public sector IT-system in its infancy can therefore prevent the discovery of usefulness problems, which have the most effect on its dependant users. It does not matter if the design is aesthetically pleasing or if the solution is efficient to use, if the user is not able to get the benefits or social services they are entitled to.

\subsection{RQ2: What are the specific challenges in recruitment of qualified candidates to usability tests in the Norwegian public sector?}
Using monetary incentives as a tool in the recruitment of participants for usability testing in the public sector can bring challenges. As described in \autoref{sec:monetary-incentives}, participation in clinical trials is used to "get a free health checkup", not being interested in the study itself. This narrative can be brought to usability testing in the public sector and its potential participants [\cite{nc_2020}]. A potential participant which is in the application process or receiving a particular benefit or social-service, might only agree to participate in a usability test with a belief that it will help them in the application process, or their benefits will be increased. Thus acting in a way during the usability test to finish it quickly, without caring about the results the participant is creating.

A country's public sector have users than span's its entire nation, which resides in large central cities or remote villages. This geographical span of users can introduce challenges when recruiting participants for a usability test. In the case with NAV, its testing lab, which mainly focuses on usability testing is located at its headquarters in Oslo [\cite{sk_2022}]. However NAV's users span then entirety of Norway, thus creating issues in recruitment as the potential candidates have to travel far in order to be able to participate in a usability test [\cite{pkf_2018}].

In the recruitment of participants for usability tests in the public sector, the recruiter must interact with the potential participant correctly or else recruitment can become difficult. As described in \autoref{sec:participant-and-researcher-interactions}, the use of scientific/technical words or phrases can lead to potential participants becoming hesitant about a research project [\cite{pkf_2018}]. This can also be the case with the public sector, as the potential participants are usually already in vulnerable life situations and might be hesitant to something they do not understand. Thus using scientific/technical language in the recruitment can turn potential participants away.

\subsection{Limitations}
A limitation in the discussed findings is that only 1 of the 18 papers for this literature review has the public sector as a concept. This could lead to the risk of the way challenges in usability testing and/or challenges in recruitment of participants in usability testing not being quite relevant for the public sector. 

Another limitation could be that only 1 of the papers discusses challenges related to recruitment of participants, and what effect this can have on usability testing. This could have the risk of the other papers on challenges in recruitment not being relevant for usability testing.

The papers reviewed in the literature was a combination of peer-review papers and conference papers. This could be seen as a limitation as conference papers are subject to change as they have not been peer-reviewed. Leading to validity issues from the results brought from the conference papers.

However with these limitations i think the arguments in the discussion is somewhat valid. The need for uncovering usability problems is the same for the public and private sector, as they both deliver complex large-scale IT-systems targeted at a diverse group of users.