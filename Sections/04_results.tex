\section{Results} \label{sec:results}
The process of searching and assessing papers as described in \autoref{sec:method} led to a total of 12 papers (\autoref{tab:relevant-papers}) which were determined to be of quality and relevant to the research questions. This section presents the results of those 12 papers.

\subsection{Challenges of performing usability tests}
From the papers relating to challenges in usability testing, five main themes were discovered: (1) Number of participants in a usability test. (2) Usability testing being used when it should not be. (3) Usability testing being used as scientific proof. (4) Problems that might arise during a usability test. And (5) Integrating usability testing in an agile software development process.

\subsubsection{Participants in usability testing} \label{sec:participants-in-usability-testing}
Choosing the correct amount of participants for a usability test can bring some challenges. When developing complex systems, it is recommended to at a minimum involve 5 participants for a usability test cycle, while up to 20 participants might be also necessary to uncover serious usability issues [\cite{ola_2019}]. However when increasing number of participants for each usability test cycle, project time and costs will increase [\cite{ola_2019}].

The greater the modification and in turn the complexity of an ePRO system, the larger the overall sample size of participants and the number of test cycles required for uncovering serious usability issues might be required [\cite{ola_2019}]. Although some suggest that there is no statistical evidence between number of participants in a usability test and number of problems found [\cite{gl_2007}]. Instead suggesting that task coverage is just as good for detecting usability problems as number of participants [\cite{gl_2007}].

Participants who are participating in an asynchronous usability test, might need more time to complete a task compared to synchronous usability tests [\cite{ola_2019}]. Thus increasing the time needed to uncover the same usability problems than in synchronous usability tests.

\subsubsection{Usability testing not always the correct tool}
It is described that: "The industry have blind faith in the use of usability testing" and "There are too many places where usability testing is incorrectly applied" [\cite{pgd_2020}]. The heavy push for usability testing as the main method for evaluating usability limits the UX professionals ability to pursue more exploratory techniques that enrich the understanding of current usability problems [\cite{pgd_2020}]. 

Alternatives to usability testing should be considered. This because if incorrectly applied it can (1) squash potentially valuable ideas early in the design process, (2) incorrectly promote poor ideas, (3) misdirect software developers into solving minor vs major problems, and (4) incorrectly suggest how a design should be adopted [\cite{sg_2008}].

Usability testings early push as an evaluation tool can hinder the design usefulness to be evaluated. Usability testing target the "to hard to use" of systems, and designs are proven effective when activities can be completed with minimal disruptive errors and the user feel satisfied when doing so. This early push hinders usefulness to be evaluated, proven by the many successful products often focus on good usability after, not before usefulness [\cite{sg_2008}].

\subsubsection{Usability testing used as science}
Usability testing can be used as a quantitative method, while it actually is a qualitative method. Hypotheses can be created for lab usability testing, however this is wrong. Hypothesis generated for the purpose of hypothesis testing is a quantitative technique to statistically make inferences about a population, based on the sample that has been tested [\cite{pgd_2020}].

Some UX researchers might report on the likelihood that a concept might be a success, based on the responses from a usability test. This is problematic as it is not statistically possible to draw a conclusion based on the opinion of about five participants in a single round of usability testing [\cite{pgd_2020}].

\subsubsection{Problems during a usability test} \label{sec:problems-during-a-usability-test}
Cognitive bias in participants who are participating in a usability test is important to consider. If participants is put into a group, they are easily influenced by those around, and might not use the system being tested as they would individually. The participants could also base their responses on other responses, or may feel the need to impress the other participants in the usability test [\cite{dn_2016}].

Other challenges related to cognitive bias that can occur during a usability test is important to consider. If the same participant is used repeatedly for testing a system, he/she will become more familiar with the system. This may be misinterpreted as the later tasks in a usability test being easier to perform than earlier tasks [\cite{dn_2016}]. Participants might also think that the moderator in the usability test is the creator of the system, feeling the need to impress or satisfying the moderator. This can lead to the participants blaming usability faults on themselves instead of the system [\cite{dn_2016}].

\subsubsection{Usability testing and agile software development}
As most of modern software development processes use some kind of agile software development as their methodology, usability testing has been introduced to some new challenges. The iterative nature of agile time-lines has made it more difficult to access new participants for usability tests [\cite{ds_2014}]. This has resulted in UX professionals performing smaller usability test for each iteration that involves less participants and less tasks than normal usability tests [\cite{ln_2012}].

UX professionals have reported their concern with validity and research ethics when working in agile iterations. The UX professionals feel that agile software development methods force them to do so few usability tests that there is a risk the tests get superficial [\cite{ln_2012}]. This lack of validity and change in how UX professionals has changed the ways they plan their usability tests, has made the UX professionals more insecure [\cite{ln_2012}].

\subsection{Challenges in recruitment to usability tests}

\subsubsection{Monetary incentives} \label{sec:monetary-incentives}
The use of monetary incentives in the recruitment of participants has to be considered. If the amount of money offered for the participation is not satisfactory due to other responsibility's, such as family commitments or full-time or part-time jobs. It will result in the potential participants not willing to invest their time to participate in research [\cite{pkf_2018}]. This is increased if the possible participants are not getting compensated properly, as participants do not attend due to concerns of losing pay or not getting leave [\cite{nc_2020}].

Monetary incentives itself should be considered, as monetary incentives as a tool for participant recruitment can lead to drop-off before a study, as the participants is mainly interested for the money [\cite{pkf_2018}]. This is also described as an issue in health related studies, where participating in clinical trails is used to "get a free health checkup". Not being interested in the study itself [\cite{nc_2020}].

\subsubsection{Logistical factors}
Some logistical factors can hinder the recruitment of potential participants. Conflicting schedules, lack of time and transport are identified frequently as obstacles for participants participating in studies [\cite{pkf_2018}]. Scheduling studies or recruitment periods during things such as, holidays for families or peak academic times for students should be avoided. As this can lead to poor recruitment, since possible participants will be busy [\cite{pkf_2018}] [\cite{nc_2020}]. 

Having in-flexible hours for participation and the proximity of where participants reside being far away from where the research takes place, can effect recruitment negatively [\cite{pkf_2018}].

\subsubsection{Participant and researcher interactions} \label{sec:participant-and-researcher-interactions}
The behaviour of potential participants in recruitment is something that can be a challenge. Many potential participants do not care to listen or understand the potential risks of participating in research [\cite{nc_2020}]. Potential participants that are less-educated might need longer discussions to comprehend and allay in their fears of consents that includes audio and video recording [\cite{nc_2020}].

Using the correct language when presenting the research project's to potential participants is important in recruitment. The use of scientific/technical words or phrases can lead to potential participants raising fear or increasing their hesitation about the research project [\cite{pkf_2018}]. 

There might be a need for a researcher to be persistent in the recruitment of participants. However should avoid acting in a way considered pushy or overbearing, as this can lead to increased hesitation in potential participants [\cite{pkf_2018}].

\subsubsection{Ineffective marketing}
Choosing the correct marketing strategy for a study is shown to be vital in the recruitment of potential participants. If choosing the wrong strategy for your target demographic of participants, it can have a negative effect on potential participants. By example, if choosing poster/flyer advertising in organisations school newsletters etc, when attempting to recruit young participants, it will have limited results [\cite{aj_2015}]. In this example using social media is a more effective marketing tool [\cite{aj_2015}].

\subsubsection{Crowdsourcing} \label{sec:crowdsourcing}
The use of crowdsourcing for the recruitment of participants for usability tests promises a large pool of potential participants. However the use of crowdsourcing can have negative effects for the quality of potential participants and decreases the researchers control over participants [\cite{kts_2010}].

Certain crowdsourcing systems anonymizes its participants identities, leading to a uncertainty in the data collected from the usability test. Participants can also try to "game" the corwdsourcing system for monetary gains, leading to similar situations as described in \autoref{sec:monetary-incentives}. This can lead to qualification testing of participants being necessary, reducing the amount of possible participants, but ensuring each participant is of quality [\cite{kts_2010}]. The qualification tests also led to a deeper understanding of each participant and why each participant acted the way they did or why thy answered what during studies [\cite{kts_2010}].